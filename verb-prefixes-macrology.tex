%!TEX root = ./verb-prefixes.tex
%%
%% Miscellaneous macrology.
%%

%% The new xparse package doesn’t have something built in to test if
%% something is either -NoValue- or just empty, but it’s easy enough to
%% implement, so here we go.
%%
%% FIXME: Detect if this is defined elsewhere, in case someone else
%% implements this obvious function.
\ExplSyntaxOn
\ProvideExpandableDocumentCommand \IfNoValueOrEmptyTF {mmm}
 	{\IfNoValueTF{#1}{#2}
		{\tl_if_empty:nTF {#1} {#2} {#3}}}
\ExplSyntaxOff

%% Italicized linguistic forms appearing in text.
%%
%% The \fm@cmd command does the underlying formatting.
%% This just calls LaTeX \emph which does the right thing, but it can be replaced.
\makeatletter
\ProvideDocumentCommand \fm@cmd { m } {\emph{#1}}
%% The \fm command is the user interface.
%%
%%  arg 1: star
%%    This wraps everything in an mbox to prevent line breaking
%%    because hyphenation can get ugly especially for suffixes.
%%  arg 2: stigma
%%    An optional argument to appear before the form. This can be a
%%    star (asterisk) since it is always a square bracket argument.
%%  arg 3: body of linguistic form
\ProvideDocumentCommand \fm { s O{} m } {%
	\IfBooleanTF{#1}%
		{\mbox{#2\fm@cmd{#3}}}
		{#2\nobreak\fm@cmd{#3}}}

%% Gloss abbreviations.
%%
%% NOTE: Requires document authors to define \glossfont usefully.
%%
%% \xx{} is used for abbreviations in text, \zz{} is used for abbreviations in
%% glosses. The latter uses a condensed font so that it takes up less space.
%%
%% Screw people who don’t like two letter macros. These are mine and I don’t
%% care one whit about your namespace.
%%
%% FIXME: Detect lack of \glossfont and print a warning.
%% FIXME: Detect lack of \scshape and print a warning. This is much harder.
\ProvideDocumentCommand \xx { m } {\textsc{#1}}
\ProvideDocumentCommand \zz { m } {{\glossfont\textsc{#1}}}

%% Slashed zero for the nul or zero symbol.
%%
%% The \texorpdfstring macro from hyperref should convert the printed
%% form (some flavour of U+0030 Digit Zero) into the real null symbol
%% character (U+2205 Empty Set) for the PDF text stream that is given
%% with cut-and-paste and in PDF bookmarks. The first argument of
%% the \texorpdfstring macro is the visual representation, the second is
%% the text stream representation. Put that in your presentation forms
%% pipe and smoke it.
%%
%% If \zerofont is undefined then we just hope for the best and use U+2205.
\@ifundefined{zerofont}%
	{\ProvideDocumentCommand \nul { } {∅}}
	%% Else \texorpdfstring is defined by the hyperref package. If not,
	%% then we just define \nul without it.
	{\@ifundefined{texorpdfstring}%
		% Just do the \zerofont.
		{\ProvideDocumentCommand \nul { } {{\zerofont 0}}}
		% Use \texorpdfstring.
		% FIXME: Using \ProvideDocumentCommand doesn’t work here. Why?
		{\newcommand{\nul}{\texorpdfstring{{\zerofont 0}}{∅}}}}

%% Root as a symbol.
%%
%% This smashes the root into a 1em vertical box, adding slight whitespace on the right.
\ProvideDocumentCommand \rtsym {} {\raisebox{0pt}[0.75em][0.25em]{√\!}}

%% Roots.
%%
%% The root symbol (U+221A Square Root) in many fonts is taller than
%% the usual cap or ascender height. This \raisebox hack sticks the
%% root into a zero-height box so that it doesn’t cause an increase
%% in leading. Since this symbol often has an excessively large right
%% sidebearing for our purposes because of the angle of the ascender,
%% the negative horizontal space in \rtkern kerns the following
%% letters a bit more closely.
%%
%% The optional argument is meant for a superscript number like ² that
%% indicates the valency of the root. The superscript is rlapped so
%% that it appears above the vee of the root symbol. Since this never
%% works properly, a \raisebox moves the superscript around to get in
%% the right spot. This can be configured using the \rtsuplift length.
%% The \raisebox for the superscript has zero height and depth so it
%% doesn't screw up other spacing.
\newlength{\rtkern}
\setlength{\rtkern}{-0.0625ex}
\newlength{\rtsuplift}
\setlength{\rtsuplift}{0.66ex}
\ProvideDocumentCommand \rt { o m } {%
	\IfNoValueOrEmptyTF{#1}{}%
	{\raisebox{\rtsuplift}[0pt][0pt]{\rlap{\unspace#1\unspace}}}%
	\raisebox{0pt}[0pt][0pt]{√}\hspace*{\rtkern}#2}
%% This command allows the root to hang on the left side. It
%% is meant for column labels where the root symbol otherwise
%% visually misaligns the column label with the rest of the column.
%% Note that the \rt{} command still applies \rtkern despite \llap.
\ProvideDocumentCommand \rtlap { m } {\llap{\rt{}}#1}

%% The Brill font doesn’t need its roots kerned.
\setlength{\rtkern}{0pt}

%% Notes for later changes.
\newlength{\dblbrackkern}
\setlength{\dblbrackkern}{-0.325ex}
\ProvideDocumentCommand \dblbrackleft {} {⟦}
\ProvideDocumentCommand \dblbrackright {} {⟧}
\ProvideDocumentCommand \FIXME { +m }
	{\mbox{\dblbrackleft}\textsc{Fixme}: #1\mbox{\dblbrackright}}

%% Example comments.
\ProvideDocumentCommand \excmt { m } {\emph{#1}}

%% Hanging right example comments.
%% 
%% This depends on \rightcomment{} in ExPex.
\ProvideDocumentCommand \exrtcmt { m } {\rightcomment{\emph{#1}}}

%% Phantom (invisible) hyphens for alignment in examples.
\ProvideDocumentCommand \· {} {\phantom{-}}

%% Phantom (invisible) equals signs for alignment in examples.
\ProvideDocumentCommand \• {} {\phantom{=}}

%% Command to force lining numerals instead of old style.
\ProvideDocumentCommand \liningnumerals {m}
	{\begingroup\addfontfeatures{Letters=Uppercase,Numbers=Lining}#1\endgroup}

%% Verb lexical entry.
\ProvideDocumentCommand \vblex { m m m m }
	{\IfNoValueOrEmptyTF{#1}{}{\fm{#1} }(\IfNoValueOrEmptyTF{#2}{}{\fm{#2}; }#3) ‘#4’}

%% Verb lexical entry in examples.
\ProvideDocumentCommand \exvblex { s m m m m }
	{\IfBooleanTF{#1}{}{\newline}\footnotesize\vblex{#2}{#3}{#4}{#5}\normalsize}

%% Binominal species names.
\ProvideDocumentCommand \species { s m m o} 
	{\textit{#2}~\textit{#3}\IfNoValueOrEmptyTF{#4}{}{~\IfNoValueTF{#1}{(}{}#4\IfNoValueTF{#1}{)}{}}}

%% Some memoir configuration for marginal notes.
\marginparmargin{right}
\setmarginnotes{1ex}{4em}{0.5ex}
%% Margin note for example number.
\DeclareDocumentCommand {\exmn} {m} {\marginpar{\footnotesize{}(#1)}}
