%!TEX root = ./verb-prefixes.tex
%!TEX encoding = UTF-8 Unicode
%%%
%%% Memoir configuration.
%%%

%%
%% Page layout.
%%

%% We will assume that the trimmed paper size is set by the class option [letterpaper].
%% If not, the following commands are necessary.
%\setstocksize{11in}{8.5in}
%\settrimmedsize{\stockheight}{\stockwidth}{*}

%% The size of the left and right margins, calculating the typeblock width automatically.
%%   Sets \spinemargin and \foremargin.
\setlrmarginsandblock{1.25in}{*}{*}

%% The size of the upper and lower margins, calculating the typeblock height automatically.
%%   Sets \uppermargin and \lowermargin.
\setulmarginsandblock{1.25in}{*}{*}

%% The distance of the header and footer from the typeblock edges.
%%   Sets \headheight and \footskip.
%%   Usually set based on \baselineskip multiples, but Memoir offers the constant value \onelineskip.
%%   Args: {<header>}{<footer>}
\setheadfoot{2\onelineskip}{2\onelineskip}

%% Amount of space to use in the upper and lower margins for the header and footer.
%%   Sets \headsep and footsep.
\setheaderspaces{*}{0.5in}{*}

%% Verify and apply the calculations.
\checkandfixthelayout

%%
%% Page style.
%%

%% The ‘mine’ page style is my default.
\copypagestyle{mine}{plain}

%% The psmarks are the marks that go in the header and footer. They are used to produce the
%% \leftmark and \rightmark macros. When a sectioning macro is used in the body, the
%% appropriate psmark is updated using these definitions below.
\makepsmarks{mine}{%
	\nouppercaseheads
	\createmark{chapter}{both}{nonumber}{}{}
	\createmark{section}{both}{shownumber}{}{. \space}
	\createmark{subsection}{both}{shownumber}{}{. \space}
	\createmark{subsubsection}{both}{shownumber}{}{. \space}
	\createmark{paragraph}{both}{shownumber}{}{. \space}
	\createplainmark{toc}{both}{\contentsname}
	\createplainmark{lof}{both}{\listfigurename}
	\createplainmark{lot}{both}{\listtablename}
	\createplainmark{bib}{both}{\bibname}
	\createplainmark{index}{both}{\indexname}
	\createplainmark{glossary}{both}{\glossaryname}}

%% How the heads should be formatted. The contents come from the \leftmark and \rightmark
%% macros produced automagically from the psmarks defined above.
%%   Args: {<pagestyle>}{<left>}{<centre>}{<right>}
\makeevenhead{mine}{}{}{\normalfont\leftmark}
\makeoddhead{mine}{}{}{\normalfont\rightmark}

%% The the rule separating the header from the body. Use a thickness of 0pt for no rule.
%%   Args: {<pagestyle>}{<span>}{<thickness>}
\makeheadrule{mine}{\textwidth}{\normalrulethickness}

%% How the feet should be formatted.
%%   Args: {<pagestyle>}{<left>}{<centre>}{<right>}
\makeevenfoot{mine}{}{\thepage}{}
\makeoddfoot{mine}{}{\thepage}{}

%% The the rule separating the footer from the body. Use a thickness of 0pt for no rule.
%%   Args: {<pagestyle>}{<span>}{<thickness>}
\makeheadrule{mine}{\textwidth}{0pt}

%%
%% Section heading style.
%%

%% The font for section headings.
\newcommand{\secstylefont}{\normalfont}

%% The style for section headings.
\setsecheadstyle{\secstylefont\Large\raggedright}
\setsubsecheadstyle{\secstylefont\large\raggedright}
\setsubsubsecheadstyle{\secstylefont\bfseries\normalsize\raggedright}

%% Skip space before section heading. Negative values prevent indentation.
\setbeforesecskip{-1ex plus -0.25ex minus -0.25ex}
\setbeforesubsecskip{-1ex plus -0.25ex minus -0.25ex}
\setbeforesubsubsecskip{-1ex plus -0.25ex minus -0.25ex}

%% Skip space after section heading. Negative values cause runin with following text.
\setaftersecskip{1ex plus 0.25ex minus 0.25ex}
\setaftersubsecskip{1ex plus 0.25ex minus 0.25ex}
\setaftersubsubsecskip{1ex plus 0.25ex minus 0.25ex}

%% The same for paragraphs and subparagraphs.
\setparaheadstyle{\normalfont\normalsize\bfseries\raggedright}
\setbeforeparaskip{-1ex plus -0.25ex minus -0.25ex}
\setafterparaskip{1ex plus 0.25ex minus 0.25ex}
\setparaindent{0pt}

\setsubparaheadstyle{\normalfont\normalsize\bfseries\raggedright}
\setbeforesubparaskip{-1ex plus -0.25ex minus -0.25ex}
\setaftersubparaskip{1ex plus 0.25ex minus 0.25ex}
\setsubparaindent{0pt}

%% Number headings all the way down to subparagraphs.
\setsecnumdepth{all}
\maxsecnumdepth{all}

%% Formatting of the heading number. Same for all headings.
%%   Uses “\the<secname>” via macro trickery.
\setsecnumformat{\csname the#1\endcsname.\quad}

%% Whether to include chapter numbers in sections.
%\renewcommand{\thesection}{\thechapter.\arabic{section}}
\renewcommand{\thesection}{\arabic{section}}

%%
%% Footnote style.
%%

\feetbelowfloat
\footmarkstyle{#1.\hfill}
\setlength{\footnotesep}{0.6\baselineskip}
\setlength{\footmarkwidth}{1.5em}
\setlength{\footmarksep}{0pt}
\setlength{\footparindent}{0pt}
\renewcommand*{\foottextfont}{\footnotesize}
\renewcommand*{\makefootmarkhook}{}
\renewcommand*{\footfudgefiddle}{64}

%% Also copy this for the thanks footnote.
\thanksmarkstyle{#1\hfill}
\setlength{\thanksmarkwidth}{1.5em}
\setlength{\thanksmarksep}{0pt}

%%
%% Table of contents, List of figures, List of tables (cft).
%%

%% Depth of ToC listing.
\maxtocdepth{subparagraph}

%% Dot to use in leaders.
\renewcommand{\cftdot}{·}

%% Width of page number box.
\setpnumwidth{2.5em}

%% Width of right margin, includes page number box.
\setrmarg{3.5em}

%% Indentation before various types of entries.
\setlength{\cftchapterindent}{0ex}
\setlength{\cftsectionindent}{0ex}
%% FIXME: How could this stuff be automatically calculated??
\setlength{\cftsubsectionindent}{2.7em}
\setlength{\cftsubsubsectionindent}{6.5em}
\setlength{\cftparagraphindent}{11.3em}
\setlength{\cftsubparagraphindent}{17.3em}

%%
%% Miscellany.
%%

%% Line (interlinear) spacing.
%%
%% To quote Peter Wilson, “Some of those that have control over the visual appearance of
%% academic theses like them to be ‘double spaced’. This, of course, will make the thesis harder
%% to read but perhaps that is the purpose, or maybe they have stock (shares) in papermills and
%% lumber companies, as the theses were usually required to be printed single sided as well.”
%% (Wilson & Madsen 2013/05/22: 48)
%%
%% \OneHalfSpacing does what most people want. \DoubleSpacing actually doubles but is
%% usually way too large. These don’t modify footnotes and floats, for which add a * to the end.
%\OneHalfSpacing
%\DoubleSpacing
\SingleSpacing
%% This modifies the actual multiplier for single spacing, which is 1 by default.
%\setSingleSpace{1}
%% To control footnote and float spacing directly, use these macros. Argument is a multiplier.
%\setPagenoteSpacing{1}
%\setFloatSpacing{1}
%% There are also environments for temporary control.
%% \begin{SingleSpace}...\end{SingleSpace}
%% \begin{Spacing}{<factor>}...\end{Spacing{<factor>}
%% \begin{OneHalfSpace}...\end{OneHalfSpace}
%% \begin{OneHalfSpace*}...\end{OneHalfSpace*}
%% \begin{DoubleSpace}...\end{DoubleSpace}
%% \begin{DoubleSpace*}...\end{DoubleSpace*}

%% Paragraph indentation.
%%
%% The Memoir class sets this to different values depending on the body typeface size and
%% also the number of columns per page. Usually this should be around 1em or 1.5em,
%% depending on the design of the typeface (narrowness of the em square, etc.).
\setlength{\parindent}{1.5em}

%% Allow the occasional empty line at the bottom of a page. This is necessary because
%% examples prevent LaTeX from fitting to the typeblock consistently.
\raggedbottom

%% Typeset short pages as ragged bottom when section headings are moved to the next page.
%% This permits chunks of empty space at the bottom so headings can start on new pages.
\raggedbottomsection

%% Interword spacing and justification.
%% Options are \fussy, \midsloppy, \sloppy. The LaTeX default is \fussy.
%% There are environments too: \begin{sloppypar}...\end{sloppypar}
%% and \begin{midsloppypar}...\end{midsloppypar}. None for fussy though.
\midsloppy

%% Tightening list spacing. Tightest is \tightlists, somewhat tight is \firmlists,
%% and normal is \defaultlists. Starred \firmlists* allows less space when
%% the list is preceded by a blank line.
%%
%% See also \zerotrivseps, \savetrivseps, and \restoretrivseps for trivlists.
\tightlists
